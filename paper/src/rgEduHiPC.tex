\documentclass[conference]{IEEEtran}
\IEEEoverridecommandlockouts
% The preceding line is only needed to identify funding in the first footnote. If that is unneeded, please comment it out.
% \usepackage{cite}
\usepackage{amsmath,amssymb,amsfonts}
\usepackage{algorithmic}
\usepackage{graphicx}
\usepackage{textcomp}
\usepackage{xcolor}
\usepackage{hyperref}
\usepackage[style=ieee,backend=biber]{biblatex}
\addbibresource{refs.bib}
\def\BibTeX{{\rm B\kern-.05em{\sc i\kern-.025em b}\kern-.08em
T\kern-.1667em\lower.7ex\hbox{E}\kern-.125emX}}
\begin{document}

\title{Dissonance Reduction Tooling for HPC Pedagogy
	% \thanks{Identify applicable funding agency here. If none, delete this.}
}

\author{\IEEEauthorblockN{1\textsuperscript{st} \href{https://orcid.org/0000-0002-2393-8056}{Rohit Goswami\hspace{1mm}\includegraphics[scale=0.06]{orcid.png}}}
	\IEEEauthorblockA{\textit{Department of Chemistry} \\
		\textit{Indian Institute of Technology Kanpur}\\
		Kanpur, India \\
		rgoswami@iitk.ac.in}
	\and
	\IEEEauthorblockN{2\textsuperscript{nd} Sonaly Goswami}
	\IEEEauthorblockA{\textit{Department of Chemistry} \\
		\textit{Indian Institute of Technology Kanpur}\\
		Kanpur, India \\
		sonaly@iitk.ac.in}
	\and
	\IEEEauthorblockN{3\textsuperscript{rd} Pranay Baldev}
	\IEEEauthorblockA{\textit{Department of Electrical Engineering} \\
		\textit{Indian Institute of Technology Kanpur}\\
		Kanpur, India \\
		bpranay@iitk.ac.in}
	\and
	\IEEEauthorblockN{4\textsuperscript{th} Shaivya Anand}
	\IEEEauthorblockA{\textit{School of Environmental Science and Technology} \\
		\textit{Indian Institute of Technology Kharagpur}\\
		Kharagpur, India \\
		shaivya.anand@iitkgp.ac.in}
	\and
	\IEEEauthorblockN{5\textsuperscript{th} \href{https://orcid.org/0000-0002-2052-0594}{Debabrata Goswami\hspace{1mm}\includegraphics[scale=0.06]{orcid.png}}}
	\IEEEauthorblockA{\textit{Department of Chemistry} \\
		\textit{Indian Institute of Technology Kanpur}\\
		Kanpur, India\\
		dgoswami@iitk.ac.in}
	% \and
	% \IEEEauthorblockN{6\textsuperscript{th} Given Name Surname}
	% \IEEEauthorblockA{\textit{dept. name of organization (of Aff.)} \\
	% 	\textit{name of organization (of Aff.)}\\
	% 	City, Country \\
	% 	email address}
}

\maketitle

\begin{abstract}
	We describe a general work-flow which scales intuitively to high-performance computing (HPC) clusters for different domains of scientific computation. We demonstrate our methodology with a radial distribution function calculation in C++, with mental models for FORTRAN and Python as well. We present a pedagogical framework for the development of guided concrete incremental techniques to incorporate domain specific knowledge and transfer existing expertise for developing high-performance, platform-independent, reproducible scientific software. This is effected by presenting the acceleration of a radial distribution function, a common algorithm in computational chemistry. Thus we assert that for domain specific algorithms, there is a language-independent pedagogical methodology which may be leveraged to ensure best practices for the scientific HPC community with minimal cognitive dissonance for practitioners and students.
\end{abstract}

\begin{IEEEkeywords}
	pedagogy, best-practices, tooling, methodology, reproducible-research
\end{IEEEkeywords}

\section{Introduction}
High performance scalable computing techniques have permeated all fields of science, engineering and technology. Digital literacy has gained traction as an invaluable tool for scientific reproducible research, with basic tools being more than adequately covered by initiatives such as the Carpentries \cite{wilsonSoftwareCarpentryGetting2006a,tealDataCarpentryWorkshops2015} workshops and certifications. In addition to the renewed focus on digital tools and their use, the community has also recognized the necessity of well developed code for research \cite{gobleBetterSoftwareBetter2014}. However, even as open workflows have been recognized by the life sciences \cite{prlicTenSimpleRules2012,altschulAnatomySuccessfulComputational2013}, high performance tools have not been addressed in terms of their unique pedagogical issues. Given that distributed computing in general is considered to be a complex topic, it is natural that students unused to computational techniques, comfortable in their own scientific domains would not be able to leverage the appropriate compute, as it has been reported that the view of a novice and an expert per domain will differ \cite{chiNatureExpertise1988}. Herein we present an overview of contemporary HPC education, and develop a methodology to incorporate best practices while facilitating familiarity by suitable generalizations and variations \cite{catramboneGeneralizingSolutionProceduresa,braithwaiteEffectsVariationPrior2015}. We have taken the domain specific example of a standard analysis code for the calculation of the radial distribution function in three dimensions \cite{frenkelUnderstandingMolecularSimulation2001}. We enumerate the nature of the algorithm in pseudo-code, as the equivalent steps in both performance oriented C++, FORTRAN, and a ``simpler'' high level language (python) to show how each may be linked to domain specific representations. Furthermore we then show the natural extension of the mental model developed for C++ to encompass concepts for the parallelism on distributed systems and thus present an optimal pedagogical perspective for guided practice \cite{ericssonRoleDeliberatePractice} of high performance computing.
\section{From Computing to High Performance Computing}
\section{Local Perspective}
% These self-contained systems are then iteratively refined towards optimal performance on HPC clusters.
\printbibliography
\end{document}
