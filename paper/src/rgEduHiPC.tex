\documentclass[conference]{IEEEtran}
\IEEEoverridecommandlockouts
% The preceding line is only needed to identify funding in the first footnote. If that is unneeded, please comment it out.
% \usepackage{cite}
\usepackage{amsmath,amssymb,amsfonts}
\usepackage{algorithmic}
\usepackage{graphicx}
\usepackage{textcomp}
\usepackage{xcolor}
\usepackage[style=ieee,backend=biber]{biblatex}
\addbibresource{refs.bib}
\def\BibTeX{{\rm B\kern-.05em{\sc i\kern-.025em b}\kern-.08em
T\kern-.1667em\lower.7ex\hbox{E}\kern-.125emX}}
\begin{document}

\title{Dissonance Reduction Tooling for HPC Pedagogy
	% \thanks{Identify applicable funding agency here. If none, delete this.}
}

\author{\IEEEauthorblockN{1\textsuperscript{st} Rohit Goswami}
	\IEEEauthorblockA{\textit{Department of Chemistry} \\
		\textit{Indian Institute of Technology Kanpur}\\
		Kanpur, India \\
		rgoswami@iitk.ac.in}
	\and
	\IEEEauthorblockN{2\textsuperscript{nd} Sonaly Goswami}
	\IEEEauthorblockA{\textit{Department of Chemistry} \\
		\textit{Indian Institute of Technology Kanpur}\\
		Kanpur, India \\
		sonaly@iitk.ac.in}
	\and
	\IEEEauthorblockN{3\textsuperscript{rd} Pranay Baldev}
	\IEEEauthorblockA{\textit{Department of Electrical Engineering} \\
		\textit{Indian Institute of Technology Kanpur}\\
		Kanpur, India \\
		bpranay@iitk.ac.in}
	\and
	\IEEEauthorblockN{4\textsuperscript{th} Shaivya Anand}
	\IEEEauthorblockA{\textit{School of Environmental Science and Technology} \\
		\textit{Indian Institute of Technology Kharagpur}\\
		Kharagpur, India \\
		shaivya.anand@iitkgp.ac.in}
	\and
	\IEEEauthorblockN{5\textsuperscript{th} Debabrata Goswami}
	\IEEEauthorblockA{\textit{Department of Chemistry} \\
		\textit{Indian Institute of Technology Kanpur}\\
		Kanpur, India\\
		dgoswami@iitk.ac.in}
	% \and
	% \IEEEauthorblockN{6\textsuperscript{th} Given Name Surname}
	% \IEEEauthorblockA{\textit{dept. name of organization (of Aff.)} \\
	% 	\textit{name of organization (of Aff.)}\\
	% 	City, Country \\
	% 	email address}
}

\maketitle

\begin{abstract}
	We describe a general work-flow which scales intuitively to high-performance computing (HPC) clusters for different domains of scientific computation. We demonstrate our methodology with a radial distribution function calculation in three languages; FORTRAN, C++ and Python. We show that incorporating appropriate tooling, namely Git, CMake, TORQUE/SLURM, and nix-language expressions for packaging into the pedagogical practice allows for high-performance, platform-independent, reproducible scientific software. For domain specific algorithms, we show that there is a language-independent pedagogical methodology which may be leveraged to ensure best practices for the scientific HPC community with minimal cognitive dissonance for practitioners and students.
\end{abstract}

\begin{IEEEkeywords}
	pedagogy, best-practices, tooling, methodology, reproducible-research
\end{IEEEkeywords}

\section{Introduction}
\section{From Computing to High Performance Computing}
\section{Local Perspective}
% These self-contained systems are then iteratively refined towards optimal performance on HPC clusters.
\printbibliography
\end{document}
